\documentclass[12pt,oneside,a4paper,brazil,english,sumario=tradicional,]{abntex2}
%%%%%%%%%%%%%%%%%%%%%%%%%%%%%%%%%%%
\usepackage[hyphenbreaks]{breakurl}
\usepackage{booktabs}
\usepackage{minted}
\usepackage{morewrites}
%\DisemulatePackage{index}
\usepackage[table]{xcolor}
%\usepackage[subfigure]{tocloft} 
\usepackage{subfig} 
\usepackage{nth}
\usepackage{pifont}
%%%%%%%%%%%%%%%%%%%%%%%%%%%%%%%%%%%
\usepackage{cmap}
\usepackage{mathptmx}
%\usepackage{lmodern}
%\usepackage{helvet}
%\renewcommand{\familydefault}{\sfdefault}
%--------------
%\renewcommand{\rmdefault}{phv} % Arial
%\renewcommand{\sfdefault}{phv} % Arial	
\usepackage[T1]{fontenc}
%\usepackage{uarial}
%\renewcommand{\familydefault}{\sfdefault}
%\usepackage{blindtext}
\usepackage[utf8]{inputenc}
\usepackage{lastpage}
\usepackage{indentfirst}
\usepackage{framed}
\usepackage{color}
\usepackage{graphicx}
\usepackage{svg}
\usepackage{amsfonts}
\usepackage{tcolorbox}
\renewcommand{\thepage}{\roman{page}}
\usepackage{hyperref}
\usepackage{epstopdf}
\usepackage[referable]{threeparttablex}
\usepackage{lipsum}
\usepackage{blindtext}
\usepackage{caption}
%\usepackage{subcaption}
\usepackage{bbm}
%\usepackage[chapter]{algorithm}
%\usepackage{algorithmic}
\usepackage{multirow}
\usepackage{rotating}
\usepackage{eurosym}
\usepackage{pdfpages}
\usepackage[acronym,toc]{glossaries}
\makeglossaries
% For loading this file the following package and lines are 
% required in main latex file:
%
%   \usepackage[acronym]{glossaries}
%   \loadglsentries{acronyms}
%   \makeglossaries
%
% The entries follows are according to the line:
% \newacronym{<label>}{<abbrv>}{<full description>}
%[A]
%\newglossaryentry{IDS}{name=IDS, description={Coordinated Universal Time}}
%\newglossaryentry{adt}{name=ADT, description={Atlantic Daylight Time}}
%\newglossaryentry{est}{name=EST, description={Eastern Standard Time}}
%\printglossaries
%\newglossaryentry{CLI}
%{
%    name=CLI,
%    description={Command Line Interface}
%}
%\newglossaryentry{XSS}
%{
%    name=XSS,
%    description={Type of computer security vulnerability typically found in Web applications}
%}
\newglossaryentry{botnet}
{
    name=botnet,
    description={inter-connected computers and controlled by a malicious user in order to launch repetitive tasks against certain target}
}
\newglossaryentry{container}
{
	name=container,
	description={Lightweight virtualization method for running multiple isolated instances of an single Linux operating system}
}
\newglossaryentry{testbed}
{
    name=TESTBED,
    description={Platform for testing of containers, applications and servers}
}
\newglossaryentry{intelligence}
{
    name=intelligence,
    description={Information previously processed by intelligence analysts and that are relevant, actionable and valuable for the organization}
}
\newglossaryentry{INTRIG}
{
    name=INTRIG,
    description={Information \& Networking Technologies Research \& Innovation Group}
}
\newglossaryentry{TEST}
{
    name=TEST,
    description={testing}
}
\newglossaryentry{malware}
{
	name=malware,
	description={This description will be write later}
}
\newglossaryentry{actionable}
{
	name=actionable,
	description={Data that must be specific enough to prompt some response, or change}
}
\newglossaryentry{REN-ISAC}
{
	name=REN-ISAC,
	description={Research and Education Networking Information Sharing and Analysis Center}
}
%%%%%%%%%%%%%%%%%%%%%%%%%%%%%%%%%%%%%%%%%%%%%
%[A]
\newacronym{AaaS}{AaaS}{ALTO-as-a-Service}
\newacronym{CDF}{CDF}{Cumulative Distribution Function}
\newacronym{ALTO}{ALTO}{Application-Layer Traffic Optimization}
\newacronym{API}{API}{Application Programming Interface}
\newacronym[\glsshortpluralkey=ASes]{AS}{AS}{Autonomous System}
\newacronym{ASN}{ASN}{Autonomous System Number}
%[B]
\newacronym{BGP}{BGP}{Border Gateway Protocol}
%[C]
\newacronym{CDN}{CDN}{Content Delivery Network}
\newacronym{CL}{CL}{Confidence Level}
\newacronym{CI}{CI}{Confidence Interval}
\newacronym{CC}{CC}{Correlation Coefficient}
%[D]
%[E]
\newacronym{EPS}{EPS}{Endpoint Property Service}
\newacronym{ECS}{ECS}{Endpoint Cost Service}
%[F]
\newacronym{FCC}{FCC}{Federal Communications Commission}
%[G]
\newacronym{GDB}{GDB}{Graph Database}
\newacronym{GTC}{GTC}{General Terms and Conditions}
%[H]
%[I]
\newacronym{IETF}{IETF}{Internet Engineering Task Force}
\newacronym{IXP}{IXP}{Internet eXchange Point}
\newacronym{ISP}{ISP}{Internet Service Provider}
\newacronym{IX.br}{IX.br}{Internet Steering Committee project in Brazil}
%[J]
%[K]
\newacronym{KP}{KP}{Knowledge Plane}
%[L]
\newacronym{LG}{LG}{Looking Glass}
%[M]
\newacronym{MBA}{MBA}{Measuring Broadband America}
%[N]
\newacronym{NAP}{NAP}{Network Access Point}
\newacronym{NOS}{NOS}{Network Operating System}
\newacronym{NSP}{NSP}{Network Service Provider}
\newacronym{NOC}{NOC}{Network Operation Center}
\newacronym{NSF}{NSF}{National Science Foundation}
%[O]
\newacronym{OS}{OS}{Operating System}
\newacronym{ODL}{ODL}{OpenDaylight}
%[P]
\newacronym{P2P}{P2P}{Peer-to-peer}
\newacronym{P4P}{P4P}{Provider Portal for Applications}
\newacronym{PID}{PID}{Provider-Defined Identifier}
%[Q]
\newacronym{QoE}{QoE}{Quality of Experience}
%[R]
\newacronym{RS}{RS}{Route Server}
\newacronym{RA}{RA}{Routing Arbiter}
%[S]
\newacronym{SDN}{SDN}{Software-Defined Networking}
\newacronym{SP}{SP}{Service Provider}
%[T]
%[U]
%[W]
\newacronym{WAN}{WAN}{Wide Area Network}
%[V]
\newacronym{VM}{VM}{Virtual Machine}
%[X]
%[Y]
%[Z]
%\usepackage[USenglish,hyperpageref]{backref} %NÚMERO DA PÁGINA DE UMA CITAÇÃO APARECENDO NAS REFERÊNCIAS 
\usepackage[alf,abnt-etal-cite=2,abnt-etal-list=0,abnt-etal-text=emph]{abntex2cite}	% 
\usepackage{unicamp}
%\graphicspath{{./eps/}}
%\DeclareGraphicsExtensions{.eps}
\newcommand{\mb}[1]{\mathbf{#1}}
\newtheorem{mydef}{Defini\c{c}\~{a}o}[chapter]
\newtheorem{lemm}{Lema}[chapter]
\newtheorem{theorem}{Teorema}[chapter]
%\floatname{algorithm}{Pseudoc\'{o}digo}
%\renewcommand{\listalgorithmname}{Lista de Pseudoc\'{o}digos}

%New packages-----------------------------------
%\usepackage[table,xcdraw]{xcolor}
\usepackage{blindtext, rotating}
\usepackage{comment}
\usepackage{breakurl}
\usepackage{multirow}
\usepackage{mathtools}
\usepackage[colorinlistoftodos]{todonotes}
\usepackage{amssymb}
\usepackage{cite}
\usepackage{hyperref}
\usepackage{url}
\usepackage[T1]{fontenc}
\usepackage{xcolor}
%\usepackage{subfig}
%\usepackage{titling}
\usepackage{pdfpages}
\usepackage{environ}
\usepackage{setspace}
\usepackage{minted}
%Trecho de código para realizar a listagem de código fonte
% Definindo novas cores
\definecolor{verde}{rgb}{0,0.5,0}
% Configurando layout para mostrar codigos C++
\usepackage{listings}
\lstset{
  language=C++,
  basicstyle=\ttfamily\small,
  keywordstyle=\color{blue},
  stringstyle=\color{verde},
  commentstyle=\color{red},
  extendedchars=true,
  showspaces=false,
  showstringspaces=false,
  numbers=left,
  numberstyle=\tiny,
  breaklines=true,
  backgroundcolor=\color{green!10},
  breakautoindent=true,
  captionpos=b,
  xleftmargin=0pt,
}

\usepackage{tabularx}


%\renewcommand{\backrefpagesname}{Citado na(s) p\'{a}gina(s):~}
%\renewcommand{\backref}{}
%\renewcommand*{\backrefalt}[4]{
%	\ifcase #1 %
%		Nenhuma cita\c{c}\~{a}o no texto.%
%	\or
%		Citado na p\'{a}gina #2.%
%	\else
%		Citado #1 vezes nas p\'{a}ginas #2.%
%	\fi}%

\definecolor{blue}{RGB}{41,5,195}
\makeatletter
\hypersetup{
     	%pagebackref=true,
		pdftitle={\@title},
		pdfauthor={\@author},
    	pdfsubject={\imprimirpreambulo},
	    pdfcreator={LaTeX with abnTeX2},
		pdfkeywords={abnt}{latex}{abntex}{abntex2}{trabalho acad\^{e}mico},
		hidelinks,					% desabilita as bordas dos links
		colorlinks=false,       	% false: boxed links; true: colored links
    	linkcolor=blue,          	% color of internal links
    	citecolor=blue,        		% color of links to bibliography
    	filecolor=magenta,      	% color of file links
		urlcolor=blue,
%		linkbordercolor={1 1 1},	% set to white
		bookmarksdepth=4
}
\makeatother
\setlength{\parindent}{2cm}

% Controle do espa\c{c}amento entre um par\'{a}grafo e outro:
\setlength{\parskip}{0.2cm}

\orientador{Prof. Dr. Christian Rodolfo Esteve Rothenberg}
\instituicao{%
    UNIVERSIDADE ESTADUAL DE CAMPINAS
    \par
    Faculdade de Engenharia El\'{e}trica e de Computa\c{c}\~{a}o	
    }
\tipotrabalho{Disserta\c{c}\~{a}o (Mestrado)}
\preambulo{Dissertation presented to the Faculty of Electrical and Computer Engineering of the University of Campinas in partial fulfillment of the requirements for the degree of Master in Electrical Engineering, in the area of Computer Engineering. \vspace{0.4cm} \\
Disserta\c{c}\~{a}o apresentada à Faculdade de Engenharia Elétrica e Computa\c{c}\~{a}o da Universidade Estadual de Campinas como parte dos requisitos exigidos para a obten\c{c}\~{a}o do título de Mestre em Engenharia Eletrica, na Àrea de Engenharia de Computa\c{c}\~{a}o.}
% --- 

%\usepackage[english]{babel}
%\usepackage[utf8]{inputenc}
\usepackage{amsmath}
\usepackage{amsfonts}
\usepackage{graphicx}
\usepackage[colorinlistoftodos]{todonotes}
\usepackage{algorithm}
\usepackage{algpseudocode}


