%%%%%%%%%%%%%%%%%%%%%%%%%%%%%%%%%%%%%%%%%%%%%%%%%%%%%%%%%%%%%%%%%%%%%%%%%%%%%%%%
\newcommand{\Author}{
Anderson dos Santos Paschoalon
}

%%%%%%%%%%%%%%%%%%%%%%%%%%%%%%%%%%%%%%%%%%%%%%%%%%%%%%%%%%%%%%%%%%%%%%%%%%%%%%%%
\newcommand{\Year}{
	Anderson dos Santos Paschoalon
}

%%%%%%%%%%%%%%%%%%%%%%%%%%%%%%%%%%%%%%%%%%%%%%%%%%%%%%%%%%%%%%%%%%%%%%%%%%%%%%%%
\newcommand{\ThesisTitle}{
SIMITAR: A Tool for Generation of Synthetic and Realistic Network Workload for Benchmarking and Testing
}

%%%%%%%%%%%%%%%%%%%%%%%%%%%%%%%%%%%%%%%%%%%%%%%%%%%%%%%%%%%%%%%%%%%%%%%%%%%%%%%%
\newcommand{\TituloDaTese}{
SIMITAR: Uma Ferramente para Geração de Trafego de Rede Sintético e Realistico para Benchmarking e Testes
}

%%%%%%%%%%%%%%%%%%%%%%%%%%%%%%%%%%%%%%%%%%%%%%%%%%%%%%%%%%%%%%%%%%%%%%%%%%%%%%%%
\newcommand{\AtaDeDefesa}{
\textbf{COMISSÃO JULGADORA - DISSERTAÇÃO DE MESTRADO}
\vspace{1cm}
\begin{flushleft}
\textbf{Candidato}: Anderson dos Santos Paschoalon \hspace{1cm}     RA: 083233 \\
\textbf{Data da Defesa}: \\
\textbf{Título da Tese}: \\
``SIMITAR: A Tool for Generation of Synthetic and Realistic Network Workload for Benchmarking and Testing''\\%english\\
``SIMITAR: Uma Ferramente para Geração de Trafego de Rede Sintético e Realistico para Benchmarking e Testes''%portuguese
\end{flushleft}
\vspace{0.2cm}
\begin{flushleft}Prof. Dr. Christian Rodolfo Esteve Rothenberg (Presidente, FEEC/UNICAMP)\\
Prof. Dr. Edmundo Roberto Mauro Madeira (IC/UNICAMP) - Membro Titular\\
Prof. Dr. Marcos Antonio de Siqueira (PADTEC) - Membro Titular
\end{flushleft}
\vspace{0.2cm} 
\begin{flushleft}Ata de defesa, com as respectivas assinaturas dos membros da Comissão Julgadora, encontra-se no processo de vida acadêmica do aluno. \end{flushleft}
}

%%%%%%%%%%%%%%%%%%%%%%%%%%%%%%%%%%%%%%%%%%%%%%%%%%%%%%%%%%%%%%%%%%%%%%%%%%%%%%%%
\newcommand{\Abstract}{

Application-Layer Traffic Optimization (ALTO) is a recently standardized protocol that provides abstract network topology and cost maps in addition to endpoint information services that can be consumed by applications in order to become network-aware and to take optimized decisions regarding traffic flows. 
In this work, we propose a public service based on the ALTO specification using public routing information available at the Brazilian Internet eXchange Points (IXPs).
Our ALTO server prototype takes the acronym AaaS (ALTO-as-a-Service) and is based on over 2.5GB of real BGP data from the 25 Brazilian IX.br public IXPs.
We evaluate our proposal in terms of functional behaviour and performance via proof-of-concept experiments, which point to the potential benefits of applications being able to take smart endpoint selection decisions when consuming the developer-friendly ALTO APIs.

\vspace{\onelineskip}

\noindent\textbf{Keywords}: Routing (Computer network management); IXPs (Internet exchange points); Computer networks; SDN (software defined networking).
}

%%%%%%%%%%%%%%%%%%%%%%%%%%%%%%%%%%%%%%%%%%%%%%%%%%%%%%%%%%%%%%%%%%%%%%%%%%%%%%%%
\newcommand{\Resumo}{
Otimização de Tráfego na Camada de Aplicação (ALTO - \textit{Application-Layer Traffic Optimization}) é um protocolo recentemente padronizado que fornece uma topologia da rede e mapa de custos abstratos, além de serviços de informação de endpoints que podem ser consumidos pelos aplicativos, a fim de tornar-se conscientes da rede e tomar decisões otimizadas sobre os  
Neste trabalho, propomos um serviço público baseado nas especificações ALTO usando informação de roteamento pública disponível nos Pontos de Troca de Tráfego (PTTs) brasileiros.
Nosso protótipo de servidor ALTO, representado pela sigla AaaS (ALTO-as-a-Service), é baseado em mais de 2,5 GB de dados BGP reais dos 25 PTTs públicos brasileiros (IX.br).
Nossa proposta é avaliada em termos de comportamento funcional e desempenho através de experimentos de prova de conceito que apontam como potencial benefício das aplicações, a capacidade de tomar decisões inteligentes na seleção de endpoint ao consumir as APIs ALTO.
\vspace{\onelineskip}
\noindent\textbf{Palavras-chaves}: Roteamento (Administração de redes de computadores); Engenharia de tráfego; Redes de computadores; Bancos de dados.

}

%%%%%%%%%%%%%%%%%%%%%%%%%%%%%%%%%%%%%%%%%%%%%%%%%%%%%%%%%%%%%%%%%%%%%%%%%%%%%%%%
\newcommand{\Acknowledgements}{
First of all, I would like to thank my advisor Prof. Dr. Christian Rothenberg for the trust and for letting me be part of his selected group of students. I would not be able to imagine the undertaking of this research without his innovative ideas, consistent support and continuous encouragement. I would therefore like to express my gratitude for being a great instructor and an even greater friend. 

At the State University of Campinas (Unicamp) I have had the opportunity to learn from the best professors. I would like to acknowledge Prof. Dr. Mauricio Magalhães, Prof. Dr. Edson Borin, Prof. Dr. Edmundo Madeira and Prof. Dr. Léo Pini for sharing their knowledge and experience, which served as a constant motivation throughout this work.

I would like to thank the committee members Marcos Antonio de Siqueira, Edmundo Roberto Mauro Madeira and Christian Rothenberg for the recommendations and insightful criticism they offered me in order to improve the final version of my thesis.

I had the good fortune to meet very talented people within the INTRIG and LCA groups. My special thanks go to Mateus, Samuel, Samira, Gyanesh, Alex, Luis, Raphael, Anderson, Hirley, Ramon, Alaelson, Claudio, Talita, Javier, Elias, Rodrigo, Roberto, Mariana, Ranyeri, Aldo, Wallace, Vitor, Raphael V. and Amadeu, Mariana and Lino for their friendship and their support throughout these two years.

I would like to express special gratitude for the financial and technical support received from the Ericsson Innovation Center (Brazil), which allowed me to carry out this research.
}

%%%%%%%%%%%%%%%%%%%%%%%%%%%%%%%%%%%%%%%%%%%%%%%%%%%%%%%%%%%%%%%%%%%%%%%%%%%%%%%%
\newcommand{\Agradecimentos}{
	exemplo	
}


%%%%%%%%%%%%%%%%%%%%%%%%%%%%%%%%%%%%%%%%%%%%%%%%%%%%%%%%%%%%%%%%%%%%%%%%%%%%%%%%
\newcommand{\Epigrafe}{
\textit{``Discipline, sooner or later, will defeat intelligence.''\\
``A disciplina cedo ou tarde vencerá a inteligência.''\\
``La disciplina tarde o temprano vencerá a la inteligencia.''}
}

%%%%%%%%%%%%%%%%%%%%%%%%%%%%%%%%%%%%%%%%%%%%%%%%%%%%%%%%%%%%%%%%%%%%%%%%%%%%%%%%
\newcommand{\EpigrafeAuthor}{
Japanese proverb
}

%%%%%%%%%%%%%%%%%%%%%%%%%%%%%%%%%%%%%%%%%%%%%%%%%%%%%%%%%%%%%%%%%%%%%%%%%%%%%%%%
\newcommand{\Dedicatoria}{

Nesssa dedicatoria, gostaria de agradescer a todos que me ajudarem por essa etapa, direta ou indiretamente. Des daqueles que me inspiraram e me motivaram a seguir por esse caminho, aqueles que me ensinaram e me ajudaram durante o processo, e a aqueles cuja simples companhia me deram energia e me motivaram para estar aqui onde estou hoje. A todos, seja os que estão listado abaixo, como aqueles cuja minha memória não me ajudou na escrita desse texto.

Gostaria de agradecer ao meu professor e orientador Christian Esteves Rothemberg, sem o qual, seja pelo ensino, seja pela orientação e apoio durante o projeto, este trabalho não teria saído do papel.

Agradeço também a todos os Intrigers, colegas de grupo e de bancada, Alex, Javier, Nathan, Daniel, Danny, Gyanesh, Rafael, Fabricio e todos os demais. Agradeço a todos os demais colegas de laboratorio do LCA, em especial a Mijail, Suelen, Amadeu, Paul, ....

Agradeço a todos os companheiros e amigos que fiz em todos esses anos de Unicamp

Agradeço a todos os grandes amigos e companheiros da Opus Dei, em especial Padre Fabiano. E também todos meus caros amigos da Igreja Batista Fonte.

Agradeço aos meus companheiros passados e atuais da casa P7: Lucas Zorzetti(Xildo), 

Agradeço a minha família, a meu Pai Tirso José Paschoalon por todo sua preocupação e ensino.  A minha Mãe Rosangela dos Santos Mota, por todo o seu carinho e amor. E a minha irmã Ariela Paschoalon, pela companhia e afeto. 

E por ultimo agradeço a Deus por todos seu dons, proteção aamor

}
